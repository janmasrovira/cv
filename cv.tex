\documentclass[11pt]{article}
\usepackage[margin=1cm]{geometry}
\usepackage{url}
\usepackage{fontawesome5}
\usepackage[x11names]{xcolor}
\usepackage{titlesec}
\usepackage[inline]{enumitem}
\usepackage{hyperref}
\usepackage{cleveref}
\hypersetup{colorlinks=true,urlcolor=RoyalBlue3,linkcolor=Salmon4,citecolor=Green4}
\usepackage{footnote}

\newcommand{\fnlink}[2]{\href{#1}{#2}\footnote{\url{#1}}}

\begin{document}

\par{\centering{\Large Jan Mas Rovira}\bigskip\par}
\titleformat{\section}{\Large\scshape\raggedright}{}{0em}{}[\titlerule]

An always up-to-date version of this CV can be found in the following link:
\begin{center}
\url{https://gitlab.com/janmasrovira/cv/-/jobs/artifacts/master/raw/cv.pdf?job=tex2pdf}
\end{center}

\vspace{0.4cm}

\begin{center}
    \hfill
    \faPhone\ +34 634551201
    \hfill
    \faEnvelope\ \href{mailto:janmasrovira@gmail.com}{janmasrovira@gmail.com}
    \hfill
    \faGlobe\ \href{https://janmasrovira.gitlab.io/ascetic-slug/}{\texttt{janmasrovira.gitlab.io/ascetic-slug}}
    \hspace*{\fill}
    \\
    \vspace{0.1cm}

    \hfill
    \faMapMarker\ Spain
    \hfill
    \faGitlab\ \href{https://gitlab.com/janmasrovira}{\texttt{gitlab.com/janmasrovira}}
    \hfill
    \faGithub\ \href{https://github.com/janmasrovira}{\texttt{github.com/janmasrovira}}
    \hspace*{\fill}
\end{center}

\section{Overview}

I am a software developer with a Master's in Computer Science and a Master's in
Mathematical Logic. My focus areas include compilers, functional programming,
and more generally the intersection of mathematics and computer science.

Currently, I am working on the Juvix compiler at Heliax. Juvix is a pure,
strongly typed, functional programming language with a focus on safety. The Juvix compiler is implemented in Haskell.
Previously, I worked as a Proof Engineer at Formal Vindications, where I applied
formal verification and logical reasoning to formally verify software.

I write about things I find interesting in my blog: \href{https://janmasrovira.gitlab.io/ascetic-slug/}{\texttt{janmasrovira.gitlab.io/ascetic-slug}}.

\section{Professional History}

\subsection*{[2021 - current] Heliax - Compiler Engineer and Language Designer}
I've worked on designing the Juvix language and implementing its compiler in Haskell.
The source code is publicly available at \href{https://github.com/anoma/juvix}{\texttt{github.com/anoma/juvix}}.

I'd like to highlight some of my contributions to the compiler:
\begin{itemize}
  \item \textbf{Type checker}. I've implemented a type checker that supports
        user defined ADTs, polymorphism, type inference and implicit arguments,
        among other features.
  \item \textbf{Termination checker}. I've implemented a termination checker based on
        structural recursion. It detects decreasing lexicographic order of
        arguments.
  \item \textbf{Intermediate languages}. I've implemented transformations
        involving intermediate languages. E.g.\ lambda lifting.
  \item \textbf{Parallelization of the pipeline}. I've parallelized the pipeline so that
        modules that don't depend on each other can be compiled in parallel.
  \item \textbf{Parser}. I've implemented a parser using parser combinators. It keeps track of source location and supports custom infix operators. The complexity of the Juvix language is comparable to Haskell.
  \item \textbf{Advanced Haskell types}. I strive to write elegant code that's
        safe and easy to maitain. I've successfully used advanced types in
        Haskell such as type families, GADT's, effects, singletons, etc.\ to
        improve the overall quality of the codebase.
\end{itemize}

\subsection*{[2020 - 2021] Formal Vindications - Formal Methods Engineer}
The aim of the company was to formally verify tachograph software. As part of
that effort, I contributed to formally verifying a fully UTC compliant (with
leap seconds) library in Rocq. The code is not publicly available.

I also implemented a literate programming framework called \texttt{Datalang}
which is used to specify and document algebraic data types. The framework
includes a compiler to OCaml, Rocq, Json Schema, Html and LaTeX. See
\href{https://gitlab.com/janmasrovira/datalang}{\texttt{gitlab.com/janmasrovira/datalang}}.

\subsection*{[2014 - 2019] Afi informàtica - Full Stack Android Developer (part time)}
I've used Java and Kotlin to write a custom application for a client. The
application was being used by the employee's of the client's company to
distribute and sell products to shops and supermarkets across Spain. The tasks
performed included design and implementation of GUIs, interaction with server
backed databases and overall knowledge of the Android system.

\section{Academic History}

\begin{tabular}{r|p{11cm}}
  \textsc{October 2020} & \textsc{Master in Pure and Applied Logic, 90 ECTS} \\
 & \emph{University of Barcelona}\\
                        & \textbf{Best student award}\footnote{The best student award was given based on grades and quality
                            of homework and projects. The award included a 1000€ bonus.} \\
\multicolumn{2}{c}{} \\

  \textsc{July 2018} & \textbf{Main (60 ECTS)} \\
                        & \textsc{Master in Innovation and Research in Informatics (Spec. Advanced Computing)} \\
                        & \emph{Polytechnic University of Catalonia} \\
                         \\
 & \textbf{Abroad (60 ECTS)}  \\
  & \textsc{Master in Algorithms, Languages and Logic} \\
                        & \emph{Chalmers University of Technology, Sweden} \\
                        \\
                        & \textbf{Overall (120 ECTS)}  \\
  \multicolumn{2}{c}{} \\

  \textsc{July 2015} & \textsc{Bachelor's Degree in Informatics Engineering, 240 ECTS} \\
               & \emph{Polytechnic University of Catalonia}\\
  \multicolumn{2}{c}{} \\
\end{tabular}

\section{Master's theses}

\begin{description}
\item[Nov 2020]
\begin{minipage}{\textwidth}

  \begin{tabular}{|p{12cm}}

    \textsc{Title}: Generalized Veltman Semantics in Agda. \\
    University of Barcelona. \\
    \fnlink{https://gitlab.com/janmasrovira/masterlogic-thesis/-/jobs/artifacts/master/raw/report.pdf?job=tex2pdf}{Written report}. \\
    \fnlink{https://gitlab.com/janmasrovira/interpretability-logics}{\texttt{Agda source code}}. \\
    \fnlink{https://gitlab.com/janmasrovira/coq-interpretability-logics}{\texttt{Rocq source code}}. \\
    \textit{Description}: This presents an Agda formalization of interpretability logics with a focus on generalized Veltman semantics.
    \\All proofs have been formalized in the Agda proof assistant. A subset of them has also been formalized in Rocq.
    \\This thesis was supervised by Joost J. Joosten.

  \end{tabular}
\end{minipage}

\item[July 2018]
  \begin{minipage}{\textwidth}
    \begin{tabular}{|p{12cm}}
      \textsc{Title}: Automatic Inductive Equational Reasoning. \\
      Polytechnic University of Catalonia. \\
      \fnlink{https://gitlab.com/janmasrovira/master-thesis-doc/-/raw/master/final-report-2.pdf?inline=false}{Written report}. \\
      \fnlink{https://gitlab.com/janmasrovira/phileas}{\texttt{Haskell source code}}. \\
      \textit{Description}: This projects presents Phileas, an automatic
      theorem prover capable of inductively proving equations on Haskell
      terms.
      \\The prover itself is implemented in Haskell.
      \\This thesis was supervised by Albert Rubio Gimeno.

    \end{tabular}
  \end{minipage}

\end{description}

\section{Publications}
\begin{description}
\item[August 2020]
  \begin{minipage}{\textwidth}
    \begin{tabular}{|p{12cm}}
      \textbf{Generalised Veltman semantics in Agda} \\
      \emph{Advances in Modal Logic 2020, Helsinki (presented online)} \\
      \textsc{Authors}: Jan Mas Rovira, Joost J. Joosten, Luka Mikec. \\
      \fnlink{https://www2.helsinki.fi/sites/default/files/atoms/files/finalshortpapermain.pdf}{Collection of the short papers accepted by AiML 2020} \\
    \end{tabular}
  \end{minipage}

\item[July 2020]
  \begin{minipage}{\textwidth}
    \begin{tabular}{|p{12cm}}
      \textbf{An overview of Generalised Veltman Semantics} \\
      \emph{Preprint} \\
      \textsc{Authors}: Joost J. Joosten, Jan Mas Rovira, Luka Mikec, Mladen Vuković. \\
      \textsc{ArXiv}: \href{https://arxiv.org/abs/2007.04722}{2007.04722} \\
    \end{tabular}
  \end{minipage}


\item[July 2017]
  \begin{minipage}{\textwidth}
    \begin{tabular}{|p{12cm}}
      \textbf{Jutge.org: Characteristics and experiences} \\
      \emph{Transactions on Learning Technologies} \\
      \textsc{Authors}: Jordi Petit (main author), Jan Mas Rovira, et al. \\
      \textsc{DOI}: \href{https://doi.org/10.1109/TLT.2017.2723389}{10.1109/TLT.2017.2723389} \\
    \end{tabular}
  \end{minipage}

\end{description}
\end{document}
