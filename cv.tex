\documentclass[a4paper,11pt]{article}

\usepackage[margin=2.9cm]{geometry}
\usepackage{xunicode,xltxtra,url,parskip}
\usepackage[usenames,dvipsnames]{xcolor}

\usepackage{hyperref}
\definecolor{linkcolour}{rgb}{0,0.2,0.6}
\hypersetup{colorlinks,breaklinks,urlcolor=linkcolour,linkcolor=linkcolour}

\usepackage{titlesec}
\titleformat{\section}{\Large\scshape\raggedright}{}{0em}{}[\titlerule]

\newcommand{\fnlink}[2]{\href{#1}{#2}\footnote{\url{#1}}}

\begin{document}

\pagestyle{empty}

\font\fb=''[cmr10]''

\par{\centering{\huge CV - Jan Mas Rovira}\bigskip\par}

\section{Personal Data}

\begin{tabular}{rl}
\textsc{Birth:} & 12th of January 1993.\\
\textsc{Phone:} & +34 634551201\\
\textsc{Email:} & \href{mailto:janmasrovira@gmail.com}{janmasrovira@gmail.com} \\
\textsc{Personal blog:} & \href{https://janmasrovira.gitlab.io/ascetic-slug/}{\texttt{janmasrovira.gitlab.io/ascetic-slug}} \\
  \textsc{Gitlab page:} & \href{https://gitlab.com/janmasrovira}{\texttt{gitlab.com/janmasrovira}}
\end{tabular}

\section{Academic History}

\begin{tabular}{r|p{11cm}}
  \textsc{October 2020} & \textsc{Master in Pure and Applied Logic, 90 ECTS} \\
 & \emph{University of Barcelona}\\
   & Grade: \textbf{7.8/10} \\
   & \textbf{Best student award} \\
\multicolumn{2}{c}{} \\

  \textsc{July 2018} & \textsc{Master in Innovation and Research in Informatics (Spec. Advanced Computing), 120 ECTS} \\
               & \emph{Polytechnic University of Catalonia with an exchange period of 1 year (60 credits) at Chalmers University of Technology, Sweden.}\\
  & Grade: \textbf{9.1/10} \\
  \multicolumn{2}{c}{} \\

  \textsc{July 2015} & \textsc{Bachelor's Degree in Informatics Engineering, 240 ECTS} \\
               & \emph{Polytechnic University of Catalonia}\\
               & Grade: \textbf{8.12/10} \\
  \multicolumn{2}{c}{} \\
\end{tabular}

\section{Professional History}
\vspace{0.2cm}
\begin{tabular}{r|p{11cm}}
  \textsc{since May 2020} & \textsc{Haskell and Coq developer} \\
  \textsc{current} & \emph{Formal Vindications S.L.} \\
  \multicolumn{2}{c}{} \\

\end{tabular}

\section{Languages}

\begin{tabular}{rl}
  \textsc{Catalan} & Native \\
  \textsc{English} & Fluent \\
  \textsc{Spanish} & Fluent \\
\end{tabular}


\newpage
\section{Projects}

\begin{description}
\item[Nov 2020]
\begin{minipage}{\textwidth}

  \begin{tabular}{|p{12cm}}

    \textsc{Master's thesis}: Generalized Veltman Semantics in Agda. \\
    University of Barcelona. \\
    \fnlink{https://gitlab.com/janmasrovira/masterlogic-thesis/-/jobs/artifacts/master/raw/report.pdf?job=tex2pdf}{Written report}. \\
    \fnlink{https://gitlab.com/janmasrovira/interpretability-logics}{\texttt{Agda source code}}. \\
    \fnlink{https://gitlab.com/janmasrovira/coq-interpretability-logics}{\texttt{Coq source code}}. \\
    \textit{Grade}: 8.2 \\
    \textit{Description}: This presents an Agda formalization of interpretability logics with a focus on generalized Veltman semantics.
    \\All proofs have been formalized in the Agda proof assistant. A subset of them has also been formalized in Coq.
    \\This project was supervised by Joost J. Joosten. \\
    \multicolumn{1}{c}{} \\

  \end{tabular}
\end{minipage}

\item[July 2018]
  \begin{minipage}{\textwidth}
    \begin{tabular}{|p{12cm}}
      \textsc{Master's thesis}: Automatic Inductive Equational Reasoning. \\
      Polytechnic University of Catalonia. \\
      \fnlink{https://gitlab.com/janmasrovira/master-thesis-doc/-/raw/master/final-report-2.pdf?inline=false}{Written report}. \\
      \fnlink{https://gitlab.com/janmasrovira/phileas}{\texttt{Haskell source code}}. \\
      \textit{Grade}: 9.5 \\
      \textit{Description}: This projects presents Phileas, an automatic
      theorem prover capable of inductively proving equations on Haskell
      terms.
      \\The prover itself is implemented in Haskell.
      \\This project was supervised by Albert Rubio Gimeno. \\
      \multicolumn{1}{c}{} \\

    \end{tabular}
  \end{minipage}

\end{description}


% \begin{tabular}{rp{12cm}}
%   \textsc{July 2018} & 
%                      & \textit{Report}: \url{https://gitlab.com/janmasrovira/masterlogic-thesis/-/jobs/artifacts/master/raw/report.pdf?job=tex2pdf} \\
%                      & \textit{Source code (Agda)}: \url{https://gitlab.com/janmasrovira/interpretability-logics} \\
%                      & \textit{Grade}: 8.2 \\
%                      & \textit{Description}: This projects presents Phileas, an automatic
%                        theorem prover capable of inductively proving equations on Haskell
%                        terms.
%   \\&The prover itself is implemented in Haskell.
%   \\&This project was supervised by Albert Rubio Gimeno. \\
%   \multicolumn{2}{c}{} \\
%   \textsc{July 2018} & \textsc{Master's thesis}: Automatic Inductive Equational Reasoning \\
%                      & \textit{Report}: \url{https://gitlab.com/janmasrovira/master-thesis-doc/blob/master/final-report.pdf} \\
%                      & \textit{Source code}: \url{https://gitlab.com/janmasrovira/phileas} \\
%                      & \textit{Grade}: 9.5 \\
%                      & \textit{Description}: This projects presents Phileas, an automatic
%                        theorem prover capable of inductively proving equations on Haskell
%                        terms.
%   \\&The prover itself is implemented in Haskell.
%   \\&This project was supervised by Albert Rubio Gimeno. \\
%   \multicolumn{2}{c}{} \\
%   \textsc{September 2017} & \textsc{Collaboration with the CakeML project}. \\
%                      &\url{https://cakeml.org/} \\
%                      &\textit{Description}: CakeML is a formally verified compiler for a substatious subset of Standard ML. I collaborated with the CakeML team at Chalmers University
%                        under the supervision of Magnus Myreen in the efforts to optimize a compilation
%                        stage of the CakeML compiler.  \\
%   \multicolumn{2}{c}{} \\
%   \textsc{July 2017} & \textsc{Co-author of a research paper}. \\
%                      &\url{https://www.researchgate.net/publication/318200476_Jutgeorg_Characteristics_and_Experiences} \\
%                      &\textit{Description}: This paper explains the experiences of using an
%                        online judge developed and used in teaching at the Polytechnic University of
%                        Catalonia. My contribution was the implementation of a Haskell code analyzer that
%                        checks some constraints on the code that the students submit. \\
%   \multicolumn{2}{c}{} \\

%   \textsc{March 2017} & \textsc{An OCaml Backend for Agda}. \\
%                      &\url{https://gitlab.com/janmasrovira/agda2mlf} \\
%                      &\textit{Description}: Initial efforts to implement an OCaml backend for
%                        the Agda programming language. Development has been continued in
%                        the official Agda repository.
%                        \\&This project was done in conjunction with Frederik H. Iversen under
%                        the supervision of Ulf Norell at Chalmers University of Technology. \\
%   \multicolumn{2}{c}{} \\

%   \textsc{July 2015} & \textsc{Bachelor's thesis}: Automatic Static Analysis of Haskell Programs. \\
%                      & \textit{Report}: (In Catalan) \url{https://upcommons.upc.edu/handle/2117/79657?locale-attribute=ca} \\
%                      & \textit{Source code}: \url{https://gitlab.com/janmasrovira/haskal} \\
%                      & \textit{Grade}: 8.8 \\
%                      & \textit{Description}: This project presents the implementation of
%                        an automatic Haskell code analyzer capable of transforming certain functions
%                        into a tail-recursive function (under some assumptions).
%                        Additionally, it can also find in some cases a strictly recursively-decreasing
%                        arithmetical expression that proves termination. \\
%                      & This project was supervised by Albert Rubio Gimeno. \\
%   \multicolumn{2}{c}{} \\
%   \textsc{-} & \textsc{Personal blog}:  \url{https://janmasrovira.gitlab.io/ascetic-slug/} \\
%                      & \textit{Description}: Personal blog where I write about topics that I find interesting. I would like to highlight the post \textit{An Agda eDSL for well-typed Hilbert style proofs}. \\

%     \multicolumn{2}{c}{} \\

% \end{tabular}

\end{document}
